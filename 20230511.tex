\documentclass{jsarticle}
% \documentclass[b4paper,landscape,14pt]{jsarticle}
\title{}
\author{}
\date{
% \number\year 年 \number\month 月
}
\usepackage{fenrir_v1_4_0}
\usepackage{ethm_v1_1_0}
\mathtoolsset{showonlyrefs=true}

\begin{document}
% \maketitle
\setcounter{section}{6}
\section{Brownian Motion and Partial Differential Equations}
\setcounter{subsection}{4}
\subsection{Planar Brownian Motion and Holomorphic Functions}
\begin{itemize}
    \item 
    $d=2$
    \item 
    $B_t=(X_t, Y_t)$: 2-dim. BM ($X, Y$: indep. real BM)
    $\xrightarrow{\real^2\text{ と }\compl\text{ を同一視}} B_t=X_t+iY_t$: \textbf{complex BM}
    \item 
    $B$: $P_z$ の下で $z$ からスタート for $\Forall z\in\compl$
\end{itemize}

$\Phi:\compl\to\compl$: 正則関数
$\implies \Re(\Phi), \Im(\Phi)$: 調和関数(by C-R関係式),$\Re(\Phi(B_t)), \Im(\Phi(B_t))$: CLM

実はより強い結果が言える.

\begin{screen}
    \setcounter{thm}{17}
    \begin{thm}\label{thm:718}~
        \begin{itemize}
            \item 
            $\Phi:\compl\to\compl$:(定数関数でない)正則関数
            \item 
            $C_t=\int_0^t\abs{\Phi'(B_s)}^2ds$ for $\Forall t\ge0$
            \item 
            $z\in\compl$
        \end{itemize}
        $\implies \Exists \Gamma$: $P_z$ の下で $\Phi(z)$ からスタートするcomplex BM s.t. $P_z$-a.s. $\Forall t\ge0, \Phi(B_t)=\Gamma_{C_t}.$
    \end{thm}
\end{screen}

言い換えると,complex BMの正則関数による像は,時間変更されたcomplex BMとなる(この性質を2-dim. BMの\textbf{共形不変性 (conformal invariance property)}という).
\begin{itemize}
    \item 
    Theorem \ref{thm:718}を $\Phi$ が領域 $D\subset \compl$ において定まり,かつ正則である場合($z\in D$ の場合)にまで拡張することが可能(多くの応用に有用).
    \item 
    $\Phi(B)^T$ についても同様の表現が成り立つ($T$: BMによる $D$ のfirst exit time).
\end{itemize}

\begin{proof}
    $\Phi:=g+ih\implies $ It\^{o}'s formulaと $g$ が調和であることより
    \begin{align}
        g(B_t)
        &= g(z)
        + \int_0^t\frac{\partial g}{\partial x}(B_s)dX_s
        + \int_0^t\frac{\partial g}{\partial y}(B_s)dY_s
        + \frac{1}{2}\int_0^t\UB{\Delta g(B_s)}{=0}ds \\
        &= g(z)
        + \int_0^t\frac{\partial g}{\partial x}(B_s)dX_s
        + \int_0^t\frac{\partial g}{\partial y}(B_s)dY_s
        \quad\text{(CLM)}.
    \end{align}
    同様に
    $$
    h(B_t)
    =h(z)
    + \int_0^t\frac{\partial h}{\partial x}(B_s)dX_s
    + \int_0^t\frac{\partial h}{\partial y}(B_s)dY_s
    $$
    もCLMとなるので,
    $$
    M_t:= g(B_t), \quad
    N_t:= h(B_t)
    $$
    と定める.
    さらにC-R関係式
    $$
    \frac{\partial g}{\partial x} = \frac{\partial h}{\partial y},\quad
    \frac{\partial g}{\partial y} = -\frac{\partial h}{\partial x}
    $$
    より $\gen{M, N}_t=0, \gen{M, M}_t=\gen{N, N}_t=C_t.$

    \begin{screen}
        $\because)$
        \begin{align}
            \gen{M, N}_t
            &= \gen{g(z)+\int_0^{\cdot}\frac{\partial g}{\partial x}(B_s)dX_s+\int_0^{\cdot}\frac{\partial g}{\partial y}(B_s)dY_s,
            h(z)+\int_0^{\cdot}\frac{\partial h}{\partial x}(B_s)dX_s+\int_0^{\cdot}\frac{\partial h}{\partial y}(B_s)dY_s}_t \\
            &= \int_0^t\frac{\partial g}{\partial x}(B_s)\frac{\partial h}{\partial x}(B_s)ds
            + \int_0^t\frac{\partial g}{\partial y}(B_s)\frac{\partial h}{\partial y}(B_s)ds \\
            &= -\int_0^t\frac{\partial g}{\partial x}(B_s)\frac{\partial g}{\partial y}(B_s)ds
            + \int_0^t\frac{\partial g}{\partial y}(B_s)\frac{\partial g}{\partial x}(B_s)ds = 0.
        \end{align}

        また,
        $$
        \Phi'
        = \frac{d\Phi}{dz}
        = \frac{\partial g}{\partial x} - i\frac{\partial g}{\partial y}
        = \frac{\partial h}{\partial y} + i\frac{\partial h}{\partial x}
        \implies \abs{\Phi'}^2
        = \abs{\nabla g}^2
        = \abs{\nabla h}^2
        $$
        であることから
        \begin{align}
            \gen{M, M}_t
            &= \gen{g(z)+\int_0^{\cdot}\frac{\partial g}{\partial x}(B_s)dX_s+\int_0^{\cdot}\frac{\partial g}{\partial y}(B_s)dY_s,
            g(z)+\int_0^{\cdot}\frac{\partial g}{\partial x}(B_s)dX_s+\int_0^{\cdot}\frac{\partial g}{\partial y}(B_s)dY_s}_t \\
            &= \int_0^t\abs{\nabla g(B_s)}^2ds
            = \int_0^t\abs{\Phi'(B_s)}^2ds
            = C_t, \\
            \gen{N, N}_t
            &= \int_0^t\abs{\nabla h(B_s)}^2ds
            = \int_0^t\abs{\Phi'(B_s)}^2ds
            = C_t.
        \end{align}
    \end{screen}

    Exercise 4.24より $P_z$-a.s.で
    $$
    \{C_\infty<\infty\}
    = \{\lim_{t\to\infty}M_t\text{ が存在し有限}\}
    = \{\lim_{t\to\infty}N_t\text{ が存在し有限}\}.
    $$
    
    一方,2-dim. BMの再帰性より,開球 $\clb$ に対し $\set{t\ge0}{B_t\in\clb}$: 非有界であるため,$\Phi(B_t)=M_t+iN_t$ は有限な極限値をもたない.
    ゆえに $P_z$-a.s.で $C_\infty=\infty.$ \\
    $\implies P_z$ の下で $M-g(z), N-h(z)$ に対しProposition 5.15の仮定
    \begin{itemize}
        \item 
        $P_z$-a.s. $\Forall t\ge0, \gen{M-g(z), M-g(z)}_t=\gen{N-h(z), N-h(z)}_t(=C_t)$
        \item 
        $P_z$-a.s. $\Forall t\ge0, \gen{M-g(z), N-h(z)}_t=0$
        \item 
        $P_z$-a.s. $\Forall t\ge0, \gen{M-g(z), M-g(z)}_{\infty}=\gen{N-h(z), N-h(z)}_{\infty}=\infty$
    \end{itemize}
    が成り立つので,0スタートのindep. real BMs $\beta, \gamma$ が存在し
    $$
    M_t=g(z)+\beta_{C_t},\quad
    N_t=h(z)+\gamma_{C_t}.
    $$

    ここで
    $$
    \Gamma_t
    := \Phi(z)+\beta_t+i\gamma_t
    $$
    と定めると,これは $\Phi(z)$ スタートのcomplex BM under $P_z$ であり,$\Forall t\ge0$ に対し
    \begin{align}
        \Gamma_{C_t}
        &= \Phi(z)+\beta_{C_t}+i\gamma_{C_t} \\
        &= \Phi(z)+(M_t-g(z))+i(N_t-h(z)) \\
        &= \Phi(z)+(M_t+iN_t)-(g(z)+ih(z)) \\
        &= \Phi(z)+\Phi(B_t)-\Phi(z) \\
        &= \Phi(B_t).
    \end{align}
\end{proof}

2-dim. BMの共形不変性を応用して,\textbf{skew-product representation}と呼ばれる2-dim. BMの極座標平面での分解を考えることができる.

\begin{screen}
    \begin{thm}[skew-product representation]\label{thm:719}~
        \begin{itemize}
            \item 
            $z\in\compl\setminus\{0\}$
            \item 
            $z:=\exp{(r+i\theta)}\ (r\in\real, \theta\in(-\pi, \pi])$
        \end{itemize}
        $\implies $ indep. linear BMs $\beta$: $r$ スタート,$\gamma$: $\theta$ スタート under $P_z$ が存在し,以下を満たす:
        
        $P_z$-a.s.で $\Forall t\ge0$ に対し
        $$
        B_t=\exp{(\beta_{H_t}+i\gamma_{H_t})},\quad
        \text{where }H_t:=\int_0^t\frac{ds}{\abs{B_s}^2}.
        $$
    \end{thm}
\end{screen}

\begin{proof}
    (この定理の「自然な」証明方法は一般化したTheorem \ref{thm:718}(共形不変性)を $\Phi(z):=\log z$ に対し適用する方法であるが,この $\Phi$ は多価関数であり,Riemann面への拡張を考える必要があるため,異なる方法で証明する.)

    $\Gamma_t:=\Gamma_t^1+i\Gamma_t^2$: $r+i\theta$ スタートのcomplex BMとする($\Gamma_t^1$: $r$ スタート,$\Gamma_t^2$: $\theta$ スタートのreal BMs).\\
    $\implies $ Theorem \ref{thm:718}を $\Phi(\xi):=e^{\xi}\ (\xi\in\compl)$ に対し適用すると
    \begin{enumerate}[label=(\roman*)]
        \item
        $\exp({\Gamma_t})=\Phi(\Gamma_t)=:Z_{C_t}$($Z$: $(\Phi(r+i\theta)=)z$ スタート)
        \item
        $C_t=\int_0^t\abs{\exp({\Gamma_s})}^2ds=\int_0^t(\exp({\Gamma_s^1}))^2\abs{\UB{\exp({i\Gamma_s^2})}{=1}}^2ds=\int_0^t\exp({2\Gamma_s^1})ds.$
    \end{enumerate}

    ここで $H_{\cdot}$ を $C_{\cdot}$ の逆関数とすると,逆関数定理 ($(f^{-1})'(f(a))=1/f'(a)$) より
    \begin{align}
        H_s
        = \int_0^s\frac{du}{C'_{H_u}}
        = \int_0^s\exp{(-2\Gamma_{H_u}^1)}du
        = \int_0^s\abs{Z_u}^{-2}du.
    \end{align}

    \begin{screen}
        $\because)$
        \begin{itemize}
            \item 
            $C_t=\int_0^t\exp({2\Gamma_s^1})ds\implies C'_{H_u}=\exp({2\Gamma_{H_u}^1})$
            \item             $\abs{Z_{C_t}}=\exp{(\Gamma_t^1)}\xrightarrow{t=H_u}\abs{Z_u}=\exp{(\Gamma_{H_u}^1)}$
        \end{itemize}
    \end{screen}

    以上より $P_z$-a.s.で $\Forall t\ge0$ に対し
    $$
    Z_t=\exp{(\Gamma_{H_t}^1+i\Gamma_{H_t}^2)},\quad
    \text{where }H_t:=\int_0^t\frac{ds}{\abs{Z_s}^2}
    $$
    が得られる.
    あとは $Z$ でなく $B$ について同様のことが言えれば結論が示せる.

    $B_t$ の偏角の連続なdetermination\nazo を $\arg B_t$ と書く(ただし $\arg B_0=\theta$).
    Theorem \ref{thm:719}の主張は
    \begin{align}
        \beta_t
        &= \log\abs{B_{\inf\set{s\ge0}{\int_0^s\abs{B_u}^{-2}du>t}}} \\
        \gamma_t
        &= \arg B_{\inf\set{s\ge0}{\int_0^s\abs{B_u}^{-2}du>t}}
    \end{align}
    $\implies \beta$: $r$ スタート,$\gamma$: $\theta$ スタートのindep. real BMs under $P_z$ であることと同値.

    \begin{screen}
        $\because)$
        $\Forall t\ge0$ に対し
        $$
        \inf\set{s\ge0}{\int_0^s\abs{B_u}^{-2}du>\int_0^t\abs{B_u}^{-2}du} = t
        $$
        であることに注意する.

        \begin{description}
            \item[$\impliedby$)] 
            \begin{align}
                \text{(RHS)}
                &= \exp{(\beta_{\int_0^t\abs{B_u}^{-2}du}
                + i\gamma_{\int_0^t\abs{B_u}^{-2}du})} \\
                &= \exp{(\log\abs{B_{\inf\set{s\ge0}{\int_0^s\abs{B_u}^{-2}du>\int_0^t\abs{B_u}^{-2}du}}}
                + i\arg B_{\inf\set{s\ge0}{\int_0^s\abs{B_u}^{-2}du>\int_0^t\abs{B_u}^{-2}du}})} \\
                &= \exp{(\log\abs{B_t}
                + i\arg B_t)} \\
                &= \abs{B_t}\exp{(i\arg B_t)} \\
                &= B_t
                = \text{(LHS)}.
            \end{align}
            \item[$\implies$)] 
            $B_t=\exp{(\beta_{\int_0^t\abs{B_u}^{-2}du})}\exp{(i\gamma_{\int_0^t\abs{B_u}^{-2}du})}=\abs{B_t}\exp{(i\arg B_t)}$ より
            \begin{itemize}
                \item 
                \begin{align}
                    \beta_{\fbox{$\int_0^t\abs{B_u}^{-2}du$}}
                    &= \log\abs{B_t} \\
                    &= \log\abs{B_{\inf\set{s\ge0}{\int_0^s\abs{B_u}^{-2}du>\fbox{$\int_0^t\abs{B_u}^{-2}du$}}}}
                \end{align}
                $\implies \beta_t=\log\abs{B_{\inf\set{s\ge0}{\int_0^s\abs{B_u}^{-2}du>t}}}.$
                \item 
                $\gamma_{\int_0^t\abs{B_u}^{-2}du} = \arg B_t
                \implies \gamma_t=\arg B_{\inf\set{s\ge0}{\int_0^s\abs{B_u}^{-2}du>t}}.$
            \end{itemize}
        \end{description}
    \end{screen}

    $\beta, \gamma$ は $B$ の決定論的な関数なので $B\beq{\text{(d)}}Z.$
    したがって $\beta+i\gamma\beq{\text{(d)}}\Gamma$ であるため所望の結果を得る.
\end{proof}

(skew-product representationに関するコメント)
$H_{\cdot}$ の逆像は $\int_0^{\cdot}\exp{(2\beta_u)}du$ と書けるので
\begin{align}
    H_{\fbox{$\int_0^t\exp{(2\beta_u)}du$}}
    = t
    = \inf\{s\ge0:\int_0^s\exp{(2\beta_u)}du>\fbox{$\int_0^t\exp{(2\beta_u)}du$}\}
\end{align}
$\implies H_t = \inf\{s\ge0:\int_0^s\exp{(2\beta_u)}du>t\}.$
$$
\therefore 
\log{\lvert B_t\rvert}
= \beta_{H_t}
= \beta_{\inf\{s\ge0:\int_0^s\exp{(2\beta_u)}du>t\}}.
$$

これより $\lvert B\rvert$ はlinear BM $\beta$ に依存して定まることがわかる.
これは $(\lvert B_t\rvert)_{t\ge0}$ がMarkov過程,i.e. 2-dim. Bessel過程であることと関係している(Bessel過程の簡単な議論はExercise 6.24やSect. 8.4.3を参照).

一方,$\theta_t:=\arg B_t=\gamma_{H_t}\implies \theta=(\theta_t)_{t\ge0}$ はMarkov過程でない.

\begin{screen}
    $\because)$(直感的な説明)
    時刻 $t$ までの $\theta$ の過去の挙動は $\lvert B_t\rvert$ の現在の値に関する情報を与える(例えば $\theta_t$ が時刻 $t$ の直前に非常に高速に振動した場合,$\lvert B_t\rvert$ は小さくなる).
    したがって過程 $\theta$ の未来の挙動に関する情報をも与えるため,Markov性が成り立たない.
\end{screen}

\end{document}
