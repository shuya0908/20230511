\documentclass{jsarticle}
% \documentclass[b4paper,landscape,14pt]{jsarticle}
\title{}
\author{}
\date{
% \number\year 年 \number\month 月
}
\usepackage{fenrir_v1_4_0}
\usepackage{ethm_v1_1_0}
\mathtoolsset{showonlyrefs=true}

\begin{document}
% \maketitle
\setcounter{section}{6}
\section{Brownian Motion and Partial Differential Equations}
\setcounter{subsection}{4}
\subsection{Planar Brownian Motion and Holomorphic Functions}
\begin{itemize}
    \item 
    $d=2$
    \item 
    $B_t=(X_t, Y_t)$: 2-dim. BM ($X, Y$: indep. real BM)
    $\xrightarrow{\real^2\text{ と }\compl\text{ を同一視}} B_t=X_t+iY_t$: \textbf{complex BM}
    \item 
    $B$: $P_z$ の下で $z$ からスタート for $\Forall z\in\compl$
\end{itemize}

$\Phi:\compl\to\compl$: 正則関数
$\implies \Re(\Phi), \Im(\Phi)$: 調和関数(by C-R関係式),$\Re(\Phi(B_t)), \Im(\Phi(B_t))$: CLM

実はより強い結果が言える.

\begin{screen}
    \setcounter{thm}{17}
    \begin{thm}\label{thm:718}~
        \begin{itemize}
            \item 
            $\Phi:\compl\to\compl$:(定数関数でない)正則関数
            \item 
            $C_t=\int_0^t\abs{\Phi'(B_s)}^2ds$ for $\Forall t\ge0$
            \item 
            $z\in\compl$
        \end{itemize}
        $\implies \Exists \Gamma$: $P_z$ の下で $\Phi(z)$ からスタートするcomplex BM s.t. $P_z$-a.s. $\Forall t\ge0, \Phi(B_t)=\Gamma_{C_t}.$
    \end{thm}
\end{screen}

言い換えると,complex BMの正則関数による像は,時間変更されたcomplex BMとなる(この性質を2-dim. BMの\textbf{共形不変性 (conformal invariance property)}という).
\begin{itemize}
    \item 
    Theorem \ref{thm:718}を $\Phi$ が領域 $D\subset \compl$ において定まり,かつ正則である場合($z\in D$ の場合)にまで拡張することが可能(多くの応用に有用).
    \item 
    $\Phi(B)^T$ についても同様の表現が成り立つ($T$: BMによる $D$ のfirst exit time).
\end{itemize}

\begin{proof}
    $\Phi:=g+ih\implies $ It\^{o}'s formulaと $g$ が調和であることより
    \begin{align}
        g(B_t)
        &= g(z)
        + \int_0^t\frac{\partial g}{\partial x}(B_s)dX_s
        + \int_0^t\frac{\partial g}{\partial y}(B_s)dY_s
        + \frac{1}{2}\int_0^t\UB{\Delta g(B_s)}{=0}ds \\
        &= g(z)
        + \int_0^t\frac{\partial g}{\partial x}(B_s)dX_s
        + \int_0^t\frac{\partial g}{\partial y}(B_s)dY_s
        \quad\text{(CLM)}.
    \end{align}
    同様に
    $$
    h(B_t)
    =h(z)
    + \int_0^t\frac{\partial h}{\partial x}(B_s)dX_s
    + \int_0^t\frac{\partial h}{\partial y}(B_s)dY_s
    $$
    もCLMとなるので,
    $$
    M_t:= g(B_t), \quad
    N_t:= h(B_t)
    $$
    と定める.
    さらにC-R関係式
    $$
    \frac{\partial g}{\partial x} = \frac{\partial h}{\partial y},\quad
    \frac{\partial g}{\partial y} = -\frac{\partial h}{\partial x}
    $$
    より $\gen{M, N}_t=0, \gen{M, M}_t=\gen{N, N}_t=C_t.$

    \begin{screen}
        $\because)$
        \begin{align}
            \gen{M, N}_t
            &= \gen{g(z)+\int_0^{\cdot}\frac{\partial g}{\partial x}(B_s)dX_s+\int_0^{\cdot}\frac{\partial g}{\partial y}(B_s)dY_s,
            h(z)+\int_0^{\cdot}\frac{\partial h}{\partial x}(B_s)dX_s+\int_0^{\cdot}\frac{\partial h}{\partial y}(B_s)dY_s}_t \\
            &= \int_0^t\frac{\partial g}{\partial x}(B_s)\frac{\partial h}{\partial x}(B_s)ds
            + \int_0^t\frac{\partial g}{\partial y}(B_s)\frac{\partial h}{\partial y}(B_s)ds \\
            &= -\int_0^t\frac{\partial g}{\partial x}(B_s)\frac{\partial g}{\partial y}(B_s)ds
            + \int_0^t\frac{\partial g}{\partial y}(B_s)\frac{\partial g}{\partial x}(B_s)ds = 0.
        \end{align}

        また,
        $$
        \Phi'
        = \frac{d\Phi}{dz}
        = \frac{\partial g}{\partial x} - i\frac{\partial g}{\partial y}
        = \frac{\partial h}{\partial y} + i\frac{\partial h}{\partial x}
        \implies \abs{\Phi'}^2
        = \abs{\nabla g}^2
        = \abs{\nabla h}^2
        $$
        であることから
        \begin{align}
            \gen{M, M}_t
            &= \gen{g(z)+\int_0^{\cdot}\frac{\partial g}{\partial x}(B_s)dX_s+\int_0^{\cdot}\frac{\partial g}{\partial y}(B_s)dY_s,
            g(z)+\int_0^{\cdot}\frac{\partial g}{\partial x}(B_s)dX_s+\int_0^{\cdot}\frac{\partial g}{\partial y}(B_s)dY_s}_t \\
            &= \int_0^t\abs{\nabla g(B_s)}^2ds
            = \int_0^t\abs{\Phi'(B_s)}^2ds
            = C_t, \\
            \gen{N, N}_t
            &= \int_0^t\abs{\nabla h(B_s)}^2ds
            = \int_0^t\abs{\Phi'(B_s)}^2ds
            = C_t.
        \end{align}
    \end{screen}

    Exercise 4.24より $P_z$-a.s.で
    $$
    \{C_\infty<\infty\}
    = \{\lim_{t\to\infty}M_t\text{ が存在し有限}\}
    = \{\lim_{t\to\infty}N_t\text{ が存在し有限}\}.
    $$
    
    一方,2-dim. BMの再帰性より,開球 $\clb$ に対し $\set{t\ge0}{B_t\in\clb}$: 非有界であるため,$\Phi(B_t)=M_t+iN_t$ は有限な極限値をもたない.
    ゆえに $P_z$-a.s.で $C_\infty=\infty.$ \\
    $\implies P_z$ の下で $M-g(z), N-h(z)$ に対しProposition 5.15の仮定
    \begin{itemize}
        \item 
        $P_z$-a.s. $\Forall t\ge0, \gen{M-g(z), M-g(z)}_t=\gen{N-h(z), N-h(z)}_t(=C_t)$
        \item 
        $P_z$-a.s. $\Forall t\ge0, \gen{M-g(z), N-h(z)}_t=0$
        \item 
        $P_z$-a.s. $\Forall t\ge0, \gen{M-g(z), M-g(z)}_{\infty}=\gen{N-h(z), N-h(z)}_{\infty}=\infty$
    \end{itemize}
    が成り立つので,0スタートのindep. real BMs $\beta, \gamma$ が存在し
    $$
    M_t=g(z)+\beta_{C_t},\quad
    N_t=h(z)+\gamma_{C_t}.
    $$

    ここで
    $$
    \Gamma_t
    := \Phi(z)+\beta_t+i\gamma_t
    $$
    と定めると,これは $\Phi(z)$ スタートのcomplex BM under $P_z$ であり,$\Forall t\ge0$ に対し
    \begin{align}
        \Gamma_{C_t}
        &= \Phi(z)+\beta_{C_t}+i\gamma_{C_t} \\
        &= \Phi(z)+(M_t-g(z))+i(N_t-h(z)) \\
        &= \Phi(z)+(M_t+iN_t)-(g(z)+ih(z)) \\
        &= \Phi(z)+\Phi(B_t)-\Phi(z) \\
        &= \Phi(B_t).
    \end{align}
\end{proof}

2-dim. BMの共形不変性を応用して,\textbf{skew-product representation}と呼ばれる2-dim. BMの極座標平面での分解を考えることができる.

\begin{screen}
    \begin{thm}[skew-product representation]\label{thm:719}~
        \begin{itemize}
            \item 
            $z\in\compl\setminus\{0\}$
            \item 
            $z:=\exp{(r+i\theta)}\ (r\in\real, \theta\in(-\pi, \pi])$
        \end{itemize}
        $\implies $ indep. linear BMs $\beta$: $r$ スタート,$\gamma$: $\theta$ スタート under $P_z$ が存在し,以下を満たす:
        
        $P_z$-a.s.で $\Forall t\ge0$ に対し
        $$
        B_t=\exp{(\beta_{H_t}+i\gamma_{H_t})},\quad
        \text{where }H_t:=\int_0^t\frac{ds}{\abs{B_s}^2}.
        $$
    \end{thm}
\end{screen}

\end{document}
