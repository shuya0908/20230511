\documentclass{jsarticle}
% \documentclass[b4paper,landscape,14pt]{jsarticle}
\title{}
\author{}
\date{
% \number\year 年 \number\month 月
}
\usepackage{fenrir_v1_4_0}
\usepackage{ethm_v1_1_0}
\mathtoolsset{showonlyrefs=true}

\begin{document}
% \maketitle
\setcounter{section}{6}
\section{Brownian Motion and Partial Differential Equations}
\setcounter{subsection}{4}
\subsection{Planar Brownian Motion and Holomorphic Functions}
\begin{itemize}
    \item 
    $d=2$
    \item 
    $B_t=(X_t, Y_t)$: 2-dim. BM
    $\xrightarrow{\real^2\text{ と }\compl\text{ を同一視}} B_t=X_t+iY_t$: \textbf{complex BM}
    \item 
    $B$: $P_z$ の下で $z$ からスタート for $\Forall z\in\compl$
\end{itemize}

$\Phi:\compl\to\compl$: 正則関数
$\implies \Re(\Phi), \Im(\Phi)$: 調和関数(by C-R関係式),$\Re(\Phi(B_t)), \Im(\Phi(B_t))$: CLM

実はより強い結果が言える.

\begin{screen}
    \setcounter{thm}{17}
    \begin{thm}\label{thm:718}~
        \begin{itemize}
            \item 
            $\Phi:\compl\to\compl$:(定数関数でない)正則関数
            \item 
            $C_t=\int_0^t\abs{\Phi'(B_s)}^2ds$ for $\Forall t\ge0$
            \item 
            $z\in\compl$
        \end{itemize}
        $\implies \Exists \Gamma$: $P_z$ の下で $\Phi(z)$ からスタートするcomplex BM s.t. $P_z$-a.s. $\Forall t\ge0, \Phi(B_t)=\Gamma_{C_t}.$
    \end{thm}
\end{screen}

言い換えると,complex BMの正則関数による像は,時間変更されたcomplex BMとなる(この性質を2-dim. BMの\textbf{共形不変性 (conformal invariance property)}という).
\begin{itemize}
    \item 
    Theorem \ref{thm:718}を $\Phi$ が領域 $D\subset \compl$ において定まり,かつ正則である場合($z\in D$ の場合)にまで拡張することが可能(多くの応用に有用).
    \item 
    $\Phi(B)^T$ についても同様の表現が成り立つ($T$: BMによる $D$ のfirst exit time).
\end{itemize}

\begin{proof}
    $\Phi:=g+ih\implies $ It\^{o}'s formulaより
    \begin{align}
        g(B_t)
        &= g(X_t,Y_t) \\
        &= 
        \begin{multlined}[t]
            g(X_0, Y_0)
            + \int_0^t\frac{\partial g}{\partial x}(X_s, Y_s)dX_s
            + \int_0^t\frac{\partial g}{\partial y}(X_s, Y_s)dY_s \\
            + \frac{1}{2}\int_0^t\frac{\partial^2 g}{\partial x^2}(X_s, Y_s)ds
            + \frac{1}{2}\int_0^t\frac{\partial^2 g}{\partial y^2}(X_s, Y_s)ds
        \end{multlined}\\
        &= g(z)
        + \int_0^t\frac{\partial g}{\partial x}(B_s)dX_s
        + \int_0^t\frac{\partial g}{\partial y}(B_s)dY_s
        + \frac{1}{2}\int_0^t\Delta g(B_s)ds \\
        &= g(z)
        + \int_0^t\frac{\partial g}{\partial x}(B_s)dX_s
        + \int_0^t\frac{\partial g}{\partial y}(B_s)dY_s
        \quad\text{(CLM)}.
    \end{align}
    同様に
    $$
    h(B_t)
    =h(z)
    + \int_0^t\frac{\partial h}{\partial x}(B_s)dX_s
    + \int_0^t\frac{\partial h}{\partial y}(B_s)dY_s
    $$
    もCLMとなるので,
    $$
    M_t:= g(B_t), \quad
    N_t:= h(B_t)
    $$
    と定める.
\end{proof}

\end{document}
